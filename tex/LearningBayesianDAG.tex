\documentclass[12pt]{article}
\usepackage[a4paper,margin=0.7in]{geometry}
\usepackage{graphicx}
\usepackage{listing}

\graphicspath{ {./img/} }
\title{Apprendere Struttura di un Grafo Bayesiano}
\author{Guglielmo Fratticioli}
\date{ Giugno 2020 }

\begin{document}

\maketitle
\begin{abstract}
Il progetto si propone di sviluppare un codice per l'apprendimento della struttura di un grafo ,
 in secondo luogo si procede al campionamento di un modello noto e quindi all'apprendimento della basato sul dataset estratto 
\end{abstract}

\section{ Sviluppo Codice }
\subsection{ Teoria }
Un modello grafico probabilistico modellizza un particolare contesto di eventi stocastici legati fra loro da dipendenze condizionali .
Si può quindi pensare ad un grafo diretto che codifica gli eventi come nodi e le dipendenze condizionali come archi fra i nodi. è importante che il grafo non sia Ciclico 
( è assurdo pensare ad un evento che condiziona se stesso ) .\\
Ad ogni particolare configurazione della rete è associata una probabilità 
congiunta degli eventi che si può fattorizzare secondo Bayes per ogni 
dipendenza condizionale, avremo quindi :  
\[P(E_1, E_2, ... E_n ) = \prod_{i=1}^n P(E_i | Pa(E_i)) = \prod_{i=1}^n \prod_{k=1}^{d_i} \prod_j P(E_i = k | Pa(E_i) = j )\]
\\Supposto che i parametri si dispongano secondo distrubuzione Multinomiale 
\[ P(E_i|\theta_1, ... \theta_{d_i}) = \theta_1^{k_1} \theta_2^{k_2} ...\ \theta_N^{k_N}  \]
\\Possiamo stimare tramite conteggio nel dataset le probabilità 
\[ P(E_i = k | Pa(E_i) = j ) = n_{ij} = \frac {\{\# \textrm{of examples where}\  E_i = k\  \textrm{while}\  Pa(E_i)\  \textrm{in config}\  j\} } { \{\# \textrm{of examples where} Pa(E_i) \textrm{in config} j\} } \]
\\assumendo che anche la Struttura sia una variabile aleatoria 
dove probabilità è totalmente concentrata nel massimo $S^\star$
\[ P(E_i | D ) = P(S^\star|D) \int P( E_i|\theta, D, S^\star )P(\theta| D,S^\star) d\theta \ \ \  \textrm{(*)} \] 
\\Secondo Bayes 
\[ P(S|D) = \frac{P(S) P(D|S)}{P(D)}\ = \]
\\Siamo interessati a massimizzare la marginial likehood $P(D|S) \propto P(S|D) $ 
Cooper e Herskoviz hanno ricavato la seguente espressione dalla relazione (*)
\[ P(D|S) \propto \prod_1^N \prod_{k=1}^{d_i} \frac{\Gamma(\alpha_{ij})}{\Gamma(\alpha_{ij} + n_{ij})} \prod_j^{q_i} \frac{\Gamma(\alpha_{ijk} + n_{ijk} )}{\Gamma(\alpha_{ijk})}\]
\\che può essere quindi usata come Scoring per una ricerca della struttura 
\subsection{ Modellizzazione }
Linguaggio : python 
\begin{itemize}
    \item Node : 
    ogni nodo memorizza una tabella dei parametri di probabilità, ha una lista di Nodi padri e Figli , 
    un etichetta ed un nome

    \item Graph :
    \begin{verbatim}
        class Graph:
            def __init__(self, nodes):
                self.nodes = nodes
    \end{verbatim}

    il grafico è memorizzato come una lista di Nodi, sono implementate varie funzioni 
    \begin{itemize}
        \item AddEdge()
        \item RemoveEdge()
        \item invertEdge()
        \item isCyclic()
    \end{itemize}

    \item Dataset : 
    
    è una lista di Esempi associati ad una lista di nodi , 
    ogni Esempio è una particolare configurazione di valori assunta dai nodi

\end{itemize}

    
\section { Campionamento }
    ho campionato gli esempi secondo  il modello ALARM secondo un ordinamento topologico, quindi si inizia dalle sorgenti : 
    \begin{verbatim}
    lvfail =  random.choices(population=[TRUE,FALSE], weights= [
        LVFAILURE.table.get('SELF')[0], LVFAILURE.table.get('SELF')[1] ] )
    
    hypovelmia = random.choices(population=[TRUE,FALSE], weights= [
        HYPOVOLEMIA.table.get('SELF')[0], HYPOVOLEMIA.table.get('SELF')[1] ] )
    \end{verbatim}
    e poi si procede a campionare i Nodi Figlio in base ai valori estratti precedentemente
    \begin{verbatim}
    lveovolume =  random.choices(population=[LOW3,NORMAL3,HIGH3],weights= [
                        LVEOVOLUME.table.get((hypovelmia[0],lvfail[0]))[0], 
                        LVEOVOLUME.table.get((hypovelmia[0],lvfail[0]))[1], 
                        LVEOVOLUME.table.get((hypovelmia[0],lvfail[0]))[2]  ] )
    \end{verbatim}


\section { Apprendimento }
    Si inizializza un DAG casuale sulle variabili e si esegue la ricerca alterando un arco nel grafo , a patto che cambi la V-STructure.
    Si inserisce quindi in una lista tutti i possibili successori validi e si sceglie quello che ha scoring migliore

\begin{verbatim}
    def Learn(graph, dataset):
        graph.initDAG()
        current = Score( graph , dataset)
        ...
        while(run):
            run = False
            vstruct = graph.VStruct()
            G = []
            S = []
            for i in range(len(graph.nodes)):
                for j in range(i, len(graph.nodes)) :
                    if i != j:
                        g1 = copy.deepcopy(graph)
                        ...
                        g1.addEdge(g1.nodes[i], g1.nodes[j])
                        g2.removeEdge(g2.nodes[i], g2.nodes[j])
                        g3.invertEdge(g3.nodes[i], g3.nodes[j])

                        if g1.VStruct() != vstruct and not g1.isCyclic():
                            S.append(Score(g1, dataset))
                            G.append(g1)
                        if g2.VStruct() != vstruct and not g2.isCyclic():
                            ...
                        ...
            if len(S) > 0:
                score = Score(G[S.index(max(S))], dataset)
                if  score > current :
                    graph = G[S.index(max(S))]
                    current = score
                    run = True
\end{verbatim}

Lo Scoring secondo Cooper e Herskoviz :
\begin{verbatim}
    def Score(graph, data):
    score = 0
    for node in graph.nodes:
        if len(node.fathers) != 0:
            j = make_j(node)
            for comb in j[0]:
                score = score + math.log(math.gamma(alphaij(node, comb))) - 
                        math.log( math.gamma(alphaij(node, comb) + 
                            Dataset.Nij(data, node, [comb, j[1]])) )
                for k in range(node.domine):
                    score = score + math.log(math.gamma(alphaijk(node, comb, k) +
                        Dataset.Nijk(data, node, [comb, j[1]], k)) - 
                           math.log(math.gamma(alphaijk(node, comb, k))))
    return score   
\end{verbatim}

\section{Results}
\includegraphics[scale = 0.3]{ALARM.png}
- The ALARM Model; Sampled with 500 examples gives Score : -4152.68

\includegraphics[scale = 0.26]{Learnt.png}
- The Learnt Structure; Learned with 500 examples gives Score : -5176.78

\end{document}
